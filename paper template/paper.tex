%!TEX program = xelatex
%!TEX TS-program = xelatex
%!TEX encoding = UTF-8 Unicode

\documentclass[algorithmlist, AutoFakeBold, AutoFakeSlant, figurelist, tablelist, nomlist, masters]{seuthesix}
\usepackage{xeCJK}
\usepackage{fontspec, xltxtra, xunicode}
\usepackage{graphicx, subfig}
\usepackage{autobreak}
\usepackage{amsmath, amssymb}
\usepackage{tabularx, array, multirow}
\usepackage{float}
\usepackage{algpseudocode}
\usepackage{booktabs}
\usepackage{enumerate}
\usepackage{longtable}
\usepackage{algorithm}
\usepackage{algorithmicx}
\usepackage{algpseudocode}
\usepackage{bm}
\usepackage{longtable}
\usepackage{enumitem}
\usepackage{natbib}

\renewcommand{\algorithmicrequire}{ \textbf{Input:}} %Use Input in the format of Algorithm
\renewcommand{\algorithmicensure}{ \textbf{Output:}} %UseOutput in the format of Algorithm


\XeTeXlinebreaklocale “zh” 
\XeTeXlinebreakskip = 0pt plus 1pt minus 0.1pt %文章内中文自动换行
%公式编号设置
% \numberwithin{equation}{section}
% % \makeatletter
% % \@addtoreset{equation}{section}
% % \makeatother
\renewcommand\theequation{\arabic{chapter}-\arabic{equation}}

\makeatletter
\newenvironment{breakablealgorithm}
  {% \begin{breakablealgorithm}
   \begin{center}
     \refstepcounter{algorithm}% New algorithm
     \hrule height.8pt depth0pt \kern2pt% \@fs@pre for \@fs@ruled
     \renewcommand{\caption}[2][\relax]{% Make a new \caption
       {\raggedright\textbf{\ALG@name~\thealgorithm} ##2\par}%
       \ifx\relax##1\relax % #1 is \relax
         \addcontentsline{loa}{algorithm}{\protect\numberline{\thealgorithm}##2}%
       \else % #1 is not \relax
         \addcontentsline{loa}{algorithm}{\protect\numberline{\thealgorithm}##1}%
       \fi
       \kern2pt\hrule\kern2pt
     }
  }{% \end{breakablealgorithm}
     \kern2pt\hrule\relax% \@fs@post for \@fs@ruled
   \end{center}
  }
\makeatother

\setcitestyle{comma}
\setlength{\bibsep}{1.5pt}
\setenumerate[1]{itemsep=0pt,partopsep=0pt,parsep=\parskip,topsep=5pt}
\setitemize[1]{itemsep=0pt,partopsep=0pt,parsep=\parskip,topsep=5pt}
\setdescription{itemsep=0pt,partopsep=0pt,parsep=\parskip,topsep=5pt}
% 表题 图题 章节号连接符
\renewcommand {\thetable} {\thechapter{}-\arabic{table}}
\renewcommand {\thefigure} {\thechapter{}-\arabic{figure}}
\begin{document}
\captionsetup{labelformat=default, labelsep=space}

% \bibliographystyle{seuthesix}
\setcounter{secnumdepth}{4}
\setcounter{tocdepth}{4}
\newtheorem{definition}{定义}[chapter]
\newcommand{\tabincell}[2]{\begin{tabular}{@{}#1@{}}#2\end{tabular}}  

\categorynumber{TP18} % 分类采用《中国图书资料分类法》
\UDC{004.8}            %《国际十进分类法UDC》的类号
\secretlevel{公开}    %学位论文密级分为"公开"、"内部"、"秘密"和"机密"四种
\studentid{181671 }   %学号要完整,前面的零不能省略。
\title{论文题目(中文)}{}{论文题目(英文)
}{}
\author{}{}
\advisor{老师名字}{职称}{老师名字(英文)}{职称(英文)}
% 空白的时候需要加转移符
% \advisor{\  }{\  }{ \ }{\  } 
% \coadvisor{楚留香}{副教授}{Perfume Tsu}{Associate Prof.} % 没有% 可以不填
\degreetype{}{Master of Engineering} % 详细学位名称
\major{}
\submajor{}
\defenddate{}
\authorizedate{\ }
\committeechair{}
\reviewer{}{}
\department{}{}
\makebigcover
\makecover

\begin{abstract}{}

\end{abstract}

\begin{englishabstract}{}


\end{englishabstract}

\tableofcontents
\mainmatter  % 该命令切换到正文状态。页码从阿拉伯数字 1 开始,此前页码为罗马数字形式。

\chapter{绪论}
\section{研究背景}



\acknowledgement

% \bibliographystyle{seuthesix}
\thesisbib{references}  % 该命令用于生成参考文献,采用 BIBTEX 工具自动生成。为此,需提供.bib 数据库文件名称作为该命令的参数,不需要包含扩展名。
\bibliographystyle{seuthesix}
% \bibliographystyle{unsrt}
%\bibliography{usa_references}

\resume{作者简介}

\end{document}
